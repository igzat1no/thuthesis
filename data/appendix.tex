% !TeX root = ../thuthesis-example.tex

\chapter{补充内容}

\section{现有可微光线追踪算法比较分析} \label{sec:compare}

在选择自行实现可微光线追踪框架前,我们调研并比较了多个现有可微光线追踪算法框架,主要包括 Mitsuba3 \cite{Mitsuba3}、Pyredner \cite{DiffMCRT}、Psdr-jit \cite{psdrjit}。下面我们对其进行简要的比较分析。

\subsection{Mitsuba3}

Mitsuba3 是由 Wenzel Jakob 等人开发的一款基于物理的渲染器,支持多种渲染算法,包括光线追踪、路径追踪、双向路径追踪等。该渲染器的计算框架基于Dr.Jit \cite{Drjit},这是一个高效的、用于数值计算和自动微分的库,使Mitsuba3能够实现可微光线追踪。这种能力使得它在计算机图形学和计算机视觉领域,特别是在需要梯度信息的优化任务中,具有极高的应用价值。

Mitsuba3 的一个显著优点是其Python接口封装得非常好,使得用户可以方便地调用各种功能。然而,尽管Mitsuba3在多方面表现出色,但其与PyTorch的兼容性却存在一定的问题。由于 Mitsuba3 基于 Dr.Jit,其与 Pytorch 并不兼容,梯度之间的传递非常困难。虽然 Mitsuba3 提供了一些可以与Pytorch兼容的方法,但是仍旧无法完全解决兼容问题。此外,非常好的封装性也意味着我们很难实现自定义模块。

我们的主要模型都是基于 Pytorch 框架实现的。我们尝试通过导出纹理图并单独使用 Mitsuba 优化,以将两者分开,然而这种方法并不高效,并且在导出纹理图时会产生误差,导致结果并不令人满意。

\subsection{Pyredner}

Redner 是由 Tzu-mao Li 等人开发的可微渲染器,Pyredner 是它的 python 版本。这个渲染器主要依照论文 Differentiable Monte Carlo Ray Tracing through Edge Sampling \cite{DiffMCRT} 实现。

Pyredner 本身依赖 Pytorch 框架进行开发,因此很容易与我们的模型兼容。Pyredner 的优点在于其简单易用,而且支持多种渲染算法,包括光线追踪、路径追踪等。然而,Pyredner 也存在一些问题,例如其文档不够完善,很多功能没有详细的说明,使得用户很难使用。关键问题在于,Pyredner 在2020年已经停止更新,这也导致其与最新版本的硬件不适配。我们也很难在 Pyredner 的基础上进行二次开发来满足我们的需求。

\subsection{Psdr-jit}

Psdr-jit是一款强大的可微渲染框架,顾名思义,这个框架同样基于Dr.Jit计算框架构建而成。它的前身是 Psdr-cuda,是一款基于物理的可微渲染器,用 C++17/CUDA 编写,并基于论文 Path-Space Differentiable Rendering \cite{PSDR} 实现。相较于其他两个渲染器,Psdr-jit 相对轻量级,同时他也支持GPU加速以及关于任意场景参数的梯度计算。

然而,与 Mitsuba3 类似,Psdr-jit同样基于 Dr.Jit 计算框架,导致梯度传递出现困难。此外,由于这是一个两人开发的小型渲染器,其说明文档并不完善,使得用户很难使用。因此我们无法选择 Psdr-jit 作为我们的基础框架。

\printbibliography
