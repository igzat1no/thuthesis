\chapter{总结与展望}

我们介绍了一种新颖的逆渲染方法,通过高质量的预处理和反复交替的可微光栅化和可微光线追踪,能够从多视角图像中重建高质量的水密流形三维网格,还原尽量真实的纹理贴图和环境光照,并主动从漫反射颜色中消除照明伪影。根据在仿真数据集上的实验,我们的方法在新视角合成和材质恢复方面表现出色。就我们所知,这项工作是目前第一个主动消除照明伪影的逆渲染方法。

我们的模型分为三个阶段。在第一阶段训练中,我们通过修改版 TensoIR,得到了三维密度场及神经网络表示的光照和材质网络。我们将其作为后续训练的高质量初始化。在第二阶段训练中,我们通过可微光栅化,将三维密度场转化为水密流形三维网格表示,并获得了与之对应的纹理和环境光照。在第三阶段训练中,我们通过可微光线追踪,优化了材质和环境光照,以消除照明伪影。我们的模型在这三个阶段的训练中,逐渐优化了三维密度场、材质和环境光照,最终得到了高质量的三维重建结果。

尽管取得部分成果,我们的方法尚有许多不足之处。首先,我们仍未在真实数据集上进行实验,因此我们的模型在真实场景中的表现尚不明确。其次,我们的模型由于引入可微光线追踪,训练速度较慢,仅仅第三阶段的训练每次大约需要二十小时。尽管在我们的任务中,这样的训练时间是可以接受的,但是在实际应用中效率也是一个重要的考量因素。再次,我们对去除照明伪影的效果仍不完全明确。我们需要设计更好的评价指标,来评估我们的模型在去除照明伪影方面的效果。如果能够按照材质对物体本身进行分割,甚至生成每个部件对应地几何结构和纹理贴图,将会使我们的模型更加完善。最后,我们的模型对物体材料和环境光照仍有较多限制。例如,我们的模型无法处理透明物体,也无法处理环境光照过强的情形。这也是由于逆渲染问题本身的不适定性。我们期望通过增加旋转光照等方式,为该问题添加更多的约束,以更好的实现还原真实材质、得到高质量三维模型的任务。

