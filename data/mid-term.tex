% !TeX root = ../thuthesis-example.tex

\chapter{中期报告}

\section{引言}

在计算机视觉和图形学领域中,逆渲染一直是一个悬而未决的重要问题。它的主要目标是从提供的多视角图像中推断出物体的三维结构和材质属性。逆渲染和传统渲染是恰好相反的两个问题。传统渲染利用场景的三维几何、材质、纹理和光照信息来模拟光线交互,从而生成给定视角下观察得到的二维图像;逆渲染则致力于从一个或多个生成的二维图像中重建物体或场景的三维几何、材质、纹理和光照信息。

如何评估逆渲染结果的好坏也是一个困难的问题。直接将重建得到的几何、材质等信息于真实信息对比是不完全合理的。这是因为在这个任务中,输入只包含从各个角度拍摄或渲染所得到的二维照片。如果一组几何、材质、光照的组合也能够在对应角度下得到与输入图像完全相同或是高相似度的图像,那么即使这个组合与真实的模型相距甚远,我们也应该承认其合理性。因此,在这个领域中,评估指标很大程度上依赖于不同视角的重渲染质量。

然而,这一间接的评价指标也就导致了问题。尽管重新渲染的图像可能具有令人印象深刻的视觉保真度,但恢复的几何、光照、材质等属性的准确性往往与真实值存在明显差异。导致这一差异的一个重要因素是,渲染的阴影和间接光照通常集成到了恢复的纹理中。这是逆渲染任务天然具有的问题,因为如果仅从图像来看,我们几乎无法区分观察到的阴影是由于光线阻挡还是源于物体材质本身就是这个颜色。

在我们的方法中,我们提出了一个新的逆渲染框架,旨在保证重渲染质量的同时恢复真实的纹理和材质细节。与采用类似 NeRF\cite{nerf} 的密度场表示不同,我们选择了三维网格表示。两者相比之下,三维网格表示更直接,也便于后续任务例如物理模拟等。为了减轻重建材质中的照明伪影,我们在我们的流程中引入了可微分光线追踪。与简单的光栅化渲染技术不同,光线追踪能够准确地模拟全局光照,从而有助于提取更真实的直接和间接光照效果。

\section{任务描述}

\subsection*{水密流形三维网格定义}

我们要求还原水密流形三维网格的几何表示,这是因为相比于密度场或是其他一些三维网格表示,这样性质良好的表示更有利于后续下游任务的进行。

\textbf{三维网格 (mesh):} 三维网格是由一系列连接的顶点、边和面构成的结构,在计算机图形学和几何建模中用于描述三维对象的形状和结构。在本工作中,我们使用三角形网格,即每个面都是一个三角形。

\textbf{水密 (watertight):} 在三维网格的每条边精确地被两个面共享且不超过两个面的情况下,该网格被称为水密的。

\textbf{流形 (manifold):} 一个流形网格需满足以下条件:(1)每条边连接的面最多为两个;(2)每条边与一个或两个面相交,与一个顶点相连的面以封闭或开放扇形结构组织;(3)面之间禁止相互自相交。

\subsection*{输入}
如上文所述,我们的任务是三维物体的逆渲染。在这个问题中,输入是一系列不同角度所拍摄或渲染的物体图像 $I_1, I_2, \cdots, I_n$,以及相对应的相机位姿参数 $p_1, p_2, \cdots, p_n$。所有图像都是通过相同的相机对同一个物体在固定光照条件下从不同角度进行拍照得到的。我们将同时使用仿真数据(通过计算机建模和渲染模拟拍照)和真实数据,以验证我们的方法的有效性。

\subsection*{输出}
我们模型的输出包含三个部分 $(G, M, L)$。其中 $G$ 代表重建的水密流形三维网格表示,$M$ 代表对应的材质信息,$L$ 则代表重建的光照。

\subsection*{评价标准}
如前所述,一个通用的评价标准是在新视角下重渲染得到的图像与真实图像的相似度。我们将使用 SSIM 和 PSNR 作为评价指标。

\textbf{SSIM} 是结构相似性指标。它考虑了亮度、对比度和结构之间的相似性,其取值范围是 $[0,1]$,其中 $1$ 表示完全相似,$0$ 表示完全不同。SSIM指标的优点之一是它对人类视觉系统的影响较好地进行了建模,因此在许多情况下能够更好地预测人类主观感知的图像质量。

\textbf{PSNR} 是峰值信噪比。它是通过直接比较原始图像和重建图像之间的像素值差异来衡量图像质量的。PSNR的计算基于均方误差(MSE),即重建图像与原始图像之间像素值的平方差的平均值。PSNR值通常以分贝(dB)为单位表示,值越高表示图像质量越好。

此外,我们的模型旨在恢复真实材质信息。因此,我们还将设计新的指标用于评估重建的材质信息与真实材质信息之间的相似度。

\section{模型设计}

上文提到我们的方法大致可以分为三个部分。首先,我们会通过 TensoIR\cite{tensoir} 得到一个密度场和对应的材质、光照等信息。其次,我们会通过可微 Marching Cube 将三维密度场转换成三维网格表示,同时通过可微光栅化重新优化材质、光照信息,以使其与三维网格表示相匹配。最后,我们通过可微光线追踪技术,固定三维网格表示,优化材质、光照信息,以去除重建的材质中的照明伪影,从而得到更真实的纹理和材质信息。下面我们将对三个部分逐一进行介绍。

\subsection{第一阶段:TensoIR}

TensoIR 是一个基于神经网络和深度学习的逆渲染框架。它的输入是一系列不同角度所拍摄或渲染的物体图像 $I_1, I_2, \cdots, I_n$,以及相对应的相机位姿参数 $p_1, p_2, \cdots, p_n$。在 TensoIR 中,我们需要同时优化密度场 $D$、外观网络 $a$ 以及环境光照 $L$。

具体而言,我们用 $D: \mathrm{R}^3 \to \mathrm{R}$ 表示密度场,$\sigma=D(x)$ 表示三维空间中坐标 $x$ 处的密度值,也可以理解为该位置被物体占据的概率。而物体的外观网络 $a$ 包含三个部分,分别是漫反射颜色 $c$,BRDF 参数 $\beta$ 和法向方向网络 $n$。其中 BRDF 参数 $\beta$ 包含两部分,分别是反射率 $\alpha = MLP_{albedo}(\beta)$ 以及表面粗糙度 $r=MLP_{roughness}(\beta)$。我们假设环境光照是一个球形高斯混合模型,即 $L(\omega) = \sum_{i=1}^k w_i \mathcal{N}(\mu_i, \Sigma_i)$,其中 $\omega$ 是光线的方向,$w_i$ 是权重,$\mu_i$ 是均值,$\Sigma_i$ 是协方差矩阵。

由于 TensoIR 同时通过神经辐射场和物理模型渲染两张图片,和真实图像计算损失函数,并同时优化上述提及的所有参数,它能够提供较为精确的密度、材质信息,可以给后续训练一个很好的初始化。

\subsection{第二阶段:可微光栅化}

在第二阶段,我们将通过可微 Marching Cube 将三维密度场转换成三维网格表示,同时通过可微光栅化重新优化材质、光照信息,以使其与三维网格表示相匹配。我们采用了类似 nvdiffrec\cite{nvdiffrec} 的训练过程。首先,我们利用 DMTet\cite{DMTet} 将 TensoIR 得出的密度场转化为三维网格表示。而 nvdiffrec 是直接从头开始训练几何表示和对应的材质等信息,相比之下我们的框架更容易训练,效果也更好。接下来,我们利用渲染器 nvdiffrast\cite{nvdiffrast} 同时优化几何、材质和光照信息。我们首先基于生成的三维网格计算光线与物体的交点位置,并利用此前得到的外观网络查询对应位置的材质信息,输入到 nvdiffrast 中进行渲染。

在 nvdiffrec 中,它直接将三维网格上存储的 SDF 值转换为网格表示。而我们的方法中只能得到密度值,并需要通过以下方式将密度值转换为这些 SDF 值:
\[ \alpha = 1 - \exp(-\sigma \cdot \delta) \]
其中,$\alpha$ 表示不透明度,$\sigma$ 表示密度值,$\delta$ 表示 TensoIR 中使用的步长。我们预先定义了一个阈值 $t$,表示表面的位置,并将 $\alpha - t$ 作为 DMTet 的输入。

\subsection{第三阶段:可微光线追踪}

在第二阶段,我们生成了一个水密的流形网格。然而,这可能会导致大量的照明伪影,因为将阴影效果“烘焙”到反照率中可以在渲染输出中创建更逼真、更连贯的场景表示。

我们利用高质量的三维网格重建和可微分光线追踪来解决这个问题。光线追踪能够几乎完全模拟光线路径,并准确计算间接照明的影响。我们高质量的几何重建进一步提高了间接光照计算的精确性,以计算逼真的阴影。传统的可微分光线追踪面临的挑战主要源于准确计算几何梯度的困难。为了解决这个问题,在第三阶段的训练中,我们通过固定几何形状,仅专注于优化材质和环境照明来解决这个问题。

我们最初选择使用现有的可微分光线追踪框架 Mitsuba\cite{Mitsuba3} 进行训练。然而,Mitsuba 及其相应的 Dr. Jit\cite{Drjit} 计算框架与 PyTorch 不兼容。我们必须将原始外观网络转换为纹理图的形式,以便导入 Mitsuba。这种转换不可避免地引入了精度损失。此外,由于第三阶段训练的结果只能转换成纹理映射的形式,直接将其转换回神经网络用于反复训练是不可行的。为了解决这个问题,我们生成了一个间接光图,并在第二阶段引入该间接光图进行训练,使第二阶段的网络逐渐靠拢第三阶段训练的结果,用这样间接的手段将第三阶段的训练结果传回第二阶段。然而,这种方法导致了更大的精度损失,并且实验结果与预期不符。

由于上述原因,我们开发了自己的可微光线追踪算法。这样我们可以直接查询外观网络和环境光网络,从而使梯度直接传播回这些网络。我们正在不断完善实现细节,并进行进一步的实验以增强其有效性。

\section{实验结果}

\subsection*{实验细节}

为了最大可能验证我们方法的有效性,我们在仿真数据和真实数据上都进行了实验。仿真数据我们采取的是重新渲染的 Nerf-Synthetic 数据,真实数据是通过 lightstage 拍摄的数据。

在第一阶段,我们使用了自己实现的 TensoIR。我们采用了 VM 分解且输入特征维数为 48 的版本。我们比较了自己实现的 TensoIR 和官方实现的实验结果,如表格 \ref{tab:tensoir} 所示。

\begin{table}[h]
  \centering
  \begin{tabular}{c|cccc}
  \textbf{物理模型的 PSNR} & lego  & hotdog & ficus & armadillo \\ \hline
  \textbf{官方版本}   & 34.7  & 36.82  & 29.78 & 39.05     \\
  \textbf{我们的实现} & 35.24 & 36.26  & 29.94 & 38.61
  \end{tabular}
  \caption{我们实现的 TensoIR 与官方版本的 TensoIR 结果比较。}
  \label{tab:tensoir}
\end{table}

对于第二阶段训练,由于显存限制,我们对输入图像做了裁剪处理。原始图像的大小是 $800 \times 800$ 像素,我们将其分成 $16$ 块,每块大小为 $200 \times 200$ 像素。我们采用了 288 分辨率的 DMTet 模型。所有实验都在一块 NVIDIA RTX 4090 GPU 上进行。

\subsection*{初步结果}

我们使用了类似于 NeRF 合成数据集的 TensoIR-synthetic 数据集。为了方便测试逆渲染模型,该数据集经过了重新渲染处理。我们还进行了仅使用我们三个阶段中的一个或两个阶段进行训练的消融研究。初步结果如表 \ref{tab:result} 所示。

\begin{table}[h]
  \centering
  \begin{tabular}{c|c|c}
                     & \textbf{hotdog} & \textbf{hotdog+lego+ficus} \\ \hline
  \textbf{nvdiffrec} & 35.2            & 29.9                       \\ \hline
  \textbf{1+2}       & 34.7            & 31.8                       \\ \hline
  \textbf{1}         & 36.8            & 33.8                       \\ \hline
  \textbf{1+2+3}     & 31.4            & 29.3                       \\ \hline
  \textbf{1+3}       & 28              & -
  \end{tabular}
\caption{TensoIR-synthetic 数据集的结果。仅使用我们模型的前两个阶段,我们在三个场景的平均 PSNR 上优于 nvdiffrec。由于我们三个阶段过程的主要目标是消除烘焙到纹理中的照明伪影,单独使用 PSNR 可能无法作为所有三个阶段的综合评估指标。事实上,我们预计引入第三阶段会降低重新渲染结果的质量。}
\label{tab:result}
\end{table}

要评估阴影去除的有效性,我们还需要设计新的用于评价 BRDF 的指标。由于我们仍在完善可微光线追踪的实现,这个特定的实验尚未完成。从当前的可视化结果来看,我们当前的方法似乎确实在减少这些伪影方面发挥了作用。

\section{结论}

我们介绍了一种新颖的逆渲染方法,用于从多视角图像中重建高质量的水密流形三维网格,以及真实的纹理贴图和环境光照。然而,我们尚未完全实现所有工作。根据当前的结果,可微光线追踪在消除照明伪影方面表现并不像预期的那样有效。鉴于光线追踪本身的复杂性,加上光照和材质之间复杂的相互作用,我们也不能很简单地定位问题来源。在未来的研究中,我们可能会探索替代方法,例如使用旋转光照来提供在不同光照条件下渲染的图像。这些增强的条件可能会为这个困难的问题带来更多的确定性。